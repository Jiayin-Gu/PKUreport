\chapter{关于此模板}




\texttt{PKUreport}是一个针对北京大学博士后研究工作报告的{\LaTeX},其制作符合\href{https://www.lib.pku.edu.cn/portal/cn/zy/dzzy/chuzhanbaogao/tijiaoyaoqiu}{北京大学博士后出站报告提交说明}。



\section*{作者}
顾加银,男,1990年出生,江苏盐城人,于2021年6月至2023年3月期间在北京大学从事博士后研究,方向为非平衡统计物理。联系方式:\texttt{gujiayin1234@163.com}。


\section*{特性}
\begin{itemize}
\item 该模板是在\texttt{book}文类的基础上加入了\texttt{ctex}宏包,支持英文和中文。关于中文的说明,可以参见 \href{https://ctan.org/pkg/ctex?lang=en}{CTeX宏集手册}。
\item 该模板遵循简单性原则,没有做过多的封装,尽量将一些格式设置放在主文件的导言区。过多的封装会隐藏模板设计细节,从而影响用户对模板的理解。这里鼓励用户根据自己的需求对模板进行修改。
\item 该模板的结构和研究报告的结构保持一致,从而可以使用户能快速且直观地理解模板的结构。
\end{itemize}



\section*{使用}
\begin{itemize}
\item 使用 \href{http://tug.org/texlive/}{TeX Live} 套装,该套装适用 Unix-like/Microsoft Windows/macOS 操作系统。
\item 使用 xelatex 编译方式。
\item 使用 biber 作为参考文献宏包 biblatex 的后台处理程序。
\end{itemize}




\section*{编译步骤}
\begin{itemize}
\item \texttt{xelatex Report.tex}
\item \texttt{biber Report}
\item \texttt{xelatex Report.tex}
\item \texttt{xelatex Report.tex}
\end{itemize}




\section*{一些具体说明}
\begin{itemize}
\item 模板的主文件是 \texttt{Report.tex},在其中可以添加章。每一章的所有相关文件放入一个子文件夹中。每一个子文件夹的路径通过 \verb"\path" 定义。
\item 摘要、致谢、文章发表分别写在相应的 \texttt{tex} 文件中。 
\item 封面信息写在 \texttt{Cover.tex} 中。
\item 原创性声明和使用授权说明是通过导入 \texttt{Declaration.pdf} 文件的形式,插入到最终的 \texttt{Report.pdf} 中的。可以单独打印 \texttt{Declaration.pdf} 文件,签字完扫描为新的文件,再导入。
\item 参考文件信息写在 \texttt{Bibliography.bib} 文件中。这里其实可以导入用户自己的文献库,通过主文件中的添加 \verb"\addbibresource{xxx.bib}" 设置完成。
\item 源文件中有时出现 \texttt{spacing} 环境,其目的是通过调整行间距,使得文字尽量充满一页,以达到美观的效果。
\end{itemize}

\section*{许可证}
该模板遵循 \href{https://choosealicense.com/licenses/gpl-3.0/}{GNU GPLv3} 许可。
