\chapter{Introduction}\label{chap:introduction}






\par The last three decades have witnessed great advance in the field of nonequilibrium statistical physics. In particular, the establishment of various relations, nowadays collectively called fluctuation theorem, have revolutionized our understanding about the dichotomy between microscopic reversibility and macroscopic irreversibility. It is now clear that the microscopic reversibility underpins these relations, whereas the macroscopic irreversibility is interpreted as an emerging property from the statistical level of description. One of these relation is the Jarzynski equality which attracts considerable interests. It is a parameter-free, model-independent relation, and allows to express the free energy difference between two equilibrium states by a nonlinear average over the required work to drive the system in a nonequilibrium process from one state to another. The other notable relation is called Gallavotti-Cohen fluctuation theorem. It is a strikingly simple and general relation which quantifies the large-deviation property of the fluctuating currents flowing across systems maintained in nonequilibrium steady state, and has already been proved in many systems. From this relation, the fluctuation-dissipation theorem, Onsager reciprocal relations and their generalizations can be easily derived.







\input{\path/Motivation.tex}
\section{Outline}

\par The purpose of the present report is to show how tensor-network approaches can be used to numerically calculate the statistics of relevant quantities in nonequilibrium systems. In fact, this nowadays constitutes one of very active subfields of statistical physics. This report is organized as follows.















